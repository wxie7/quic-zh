\section{加密和传输握手(Cryptographic and Transport Handshake)}
QUIC依赖结合了加密和传输握手的方法去最小化连接建立的时延。QUIC使用
CRYPTO帧(\ref{subsec:crypto-frames})去传输加密的握手包。本文档中定义
的QUIC版本被标识为0x00000001并且使用
[\ref{subsec:normative-references:quic-tls}]中描述的TLS;不同的QUIC版
本表明使用了不同的加密握手协议。

QUIC提供可靠的、按序到达的加密握手数据。QUIC包保护(packet protection)
被用来加密尽可能多的握手协议。加密握手必须(MUST)提供以下属性:
\begin{itemize}
\item 认证密钥交换,其中
  \begin{itemize}
  \item 服务端总是被认证
  \item 客户端可选择地被认证
  \item 每次连接都不产生不同的、不相关的密钥
  \item 密钥可被用来对0-RTT、1-RTT包的包保护
  \end{itemize}
\item 两个端点的认证的传输参数交换和对服务端的传输参数的加密保护(前往
  \ref{subsec:transport-parameters})。
\item 应用层协议的认证协商(TLS为此使用了应用层协议协商(ALPN)
  [\ref{subsec:informative-references:alpn}])。
\end{itemize}

CRYPTO帧可以在不同的包(packet)序号空间中发送
(\ref{subsec:packet-numbers})。CRYPTO帧使用的偏移量确保加密的握手数
据的有序交付从每个包序号空间的0开始。

\subsection{传输参数}
\label{subsec:transport-parameters}
