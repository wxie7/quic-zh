\section{数据报(Datagram)大小}
\subsection{初始的数据报大小}
\label{subsec:initial-datagram-size}
一个UDP数据报可以包含一个或多个QUIC包。数据报的大小指的是单个UDP数据报
的有效载荷(payload)能够携带的QUIC包的大小的总和。数据报的大小包括一
个或多个QUIC包的头和被保护的载荷,但是不包含UDP或IP的头。

数据报的最大大小被定义为单个UDP数据包能够穿过网络路径发送的最大大小。
QUIC不能(MUST NOT)在不支持最少1200bytes的数据包的网络中使用。

QUIC假设最小的IP报文的大小为1280字节,这是最小的IPv6报文的大小,并且被
大多数现代IPv4网络支持。假设IPv6的报文头最小是40bytes、IPv4的报文头最
小是20bytes、UDP的报文头最小是8bytes,这导致IPv6的最大数据报的大小是
1232bytes、IPv4的最大数据报的大小是1252bytes。因此,现代的IPv4和IPv6网
络应该能够支持QUIC。

\begin{quote}
  如果路径仅支持1280bytes的IPv6最小MTU,则支持1200字节UDP有效负载的要
  求将IPv6扩展头的可用空间限制为32bytes,或将IPv4选项的可用空间限制为
  52bytes。这会影响初始数据包和路径验证。
\end{quote}

最大数据报的大小超过1200bytes的可以通过路径最大传输单元发现(PMTUD)
(前往\ref{subsubsec:handling-of-icmp-messages-by-pmtud})或数据报打包
层PMTU发现(DPLMPTUD)(前往
\ref{subsec:datagram-packetization-layer-pmtu-discovery})来发现。

\subsection{初始数据报大小}
\subsection{路径最大传输单元}
\subsubsection{通过PMTUD处理ICMP报文}
\label{subsubsec:handling-of-icmp-messages-by-pmtud}
\subsection{数据报打包层发现}
\label{subsec:datagram-packetization-layer-pmtu-discovery}