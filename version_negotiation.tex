\section{Version Negotiation(版本协商)}
版本协商允许服务器去指明它不支持的,但客户端使用的版本。服务器对每个收
到的可能用来初始化一个新的连接的包发送一个版本协商的包作为响应。前往
\ref{subsec:match-packet-connection}查看细节。

客户端发送的第一个数据包的大小将确定服务器是否发送版本协商包。支持多个
QUIC版本的客户端应该确保(SHOULD)它们发送的第一个UDP数据报的大小是它
们所支持的所有版本的最小数据报大小的最大值,必要时使用PADDING帧
(\ref{subsec:padding-frames})。这可以确保服务器响应它和客户端相互支
持的版本。如果服务器接收到的数据报小于在其他版本中指定的最小大小,则服
务器可能不会发送版本协商包;前往\ref{subsec:initial-datagram-size}查看。

\subsection{发送版本协商包}
如果客户端选择的版本不能被服务器接受,服务器将使用版本协商数据包进行响
应;前往\ref{subsubsec:version-negotiation-packet}查看。服务器发送的数
据包中包含了服务器会接受的一系列版本。端点在接到版本协商包不能(MUST
NOT)再发送版本协商包作为响应。

该系统允许服务器处理具有不支持版本的数据包,而无需保留状态。尽管
Initial包和Version Negotiation包作为响应发送时都有可能丢失,但客户端会 
发送新的包直到它成功地接受响应或者被禁止连接。

一个服务器可能(MAY)会限制它发送的版本协商包的数量。例如,能够
识别0-RTT包的服务器在响应0-RTT数据包时可能选择不发送版本协商数据包,
并期望最终收到一个Initial包。

\subsection{处理版本协商包}
版本协商包被设计用来在未来协商用于QUIC连接的QUIC版本。 未来的Standards
Track规范可能会改变支持多个版本的 QUIC 的实现如何对接收到的版本协商数
据包作出反应,以尝试使用这个版本建立连接。

一个仅支持当前版本的QUIC的客户端在接到版本协商包时必须(MUST)放弃当前
的连接尝试,以下两种情况是例外。

\begin{itemize}
\item 如果客户端已经接收并成功处理了任何其他数据包,包括早期的版本协商
  包,则必须(MUST)丢弃任何版本协商包。
\item 客户端必须(MUST)丢弃列出了被客户端选择的QUIC版本的版本协商包。
\end{itemize}
% FIXME
如何实现版本协商被未来的Standards Track规范列为未来的工作。特别的,未
来的工作将会确保QUIC协议在面对版本下降攻击时的鲁棒性;前往
\ref{subsec:version-downgrade} 查看。